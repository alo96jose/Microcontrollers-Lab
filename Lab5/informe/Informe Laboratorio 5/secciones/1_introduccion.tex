% introduccion/resumen

%% consiste en el resumen del desarrollo y las conclusiones más importantes
La práctica para este laboratorio se centra en el uso del Arduino Nano 33 BLE y TinyML para el reconocimiento de actividad humana (HAR). Con el estudio de conceptos relacionados con la inteligencia artificial y Machine Learning em sistemas embedidos, se busca utilizar sensores de movimiento para capturar datos que posteriormente son utilizados para entrenar modelos capaces de identificar diferentes tipos de actividades. Se utiliza Arduino IDE para escribir el código y cargar el programa a la placa, con Edge Impulse se crea y entrena el modelo de machine learning (TinyML) con los datos recogidos por el Arduino para lograr el reconocimiento de actividad humana, y para leer/almacenar los datos enviados por el Arduino a la computadora se usa el serial monitor. \\

%conclusion
Se logra crear un programa el cual detecta distintos movimientos como flexión, golpe o un círculo al mover el arduino, esto mediante los sensores de giroscopio y el acelerómetro. Este tipo de programas es muy utilizado en tecnología de industria y comercial, tal es el ejemplo de los controles de consolas para videojuegos los cuales tienen un sistema similar que detecta e identifica movimientos realizados con dichos controles. \\ 

%link github 
En el siguiente link se encuentra el repositorio de trabajo: \url{https://github.com/NisseUR/IE-0624_Lab5}

