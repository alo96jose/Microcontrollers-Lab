% Conclusiones y recomendaciones

\begin{itemize}
    \item Se logró capturar la información proveniente del giroscopio en la placa del Arduino, la cual pasó por el puerto serial y se guardó a través de un script de Python en un archivo .csv.
    \item  Se utilizó de forma exitosa un script de Python en Google Colab para cargar las muestras, configurar la red neuronal, entrenar la red y la exportación del modelo.h; modelo obtenido con TensorFlow Lite.
    \item Se logró entrenar una red neuronal computacional con el 60 \% de los datos de las muestras, se utilizó el 20\% de los mismos para lo que fue la validación y 20\% restante para la realización de pruebas.
    \item Se lograron coincidencias del casi 100\% en los movimientos detectados por el algoritmo que hizo uso del modelo de TensorFlow Lite exportado, es decir, todos los movimientos puestos a prueba con el modelo fueron acertados.
    \item Tomar el Arduino Nano BLE Sense Lite siempre en la misma posición y siguiendo de ejemplo las imágenes en sección desarrollo, para obtener eficiencias de detección de movimiento cercanas al 100\%.
     \item No utilizar Arduino IDE para subir el programa y el modelo que clasifica los movimientos, puesto que este desconecta el puerto serial y no permite que se suba el programa a la placa del Arduino.
    \item La librería Arduino\_TensorFlowLite no funcionará en Windows si no mueve la carpeta tensor tensorflow dentro de la carpeta src dentro de la misma librería.
    \item Utilizar arduino-cli para compilar y cargar el programa en caso de que el Arduino IDE esté dando problema de desconexión del puerto serial a la hora de subir el programa a la placa de Arduino.
    \item Hacer uso de arduino-cli para evitar inconvencias a la hora de monitorear los datos del puerto serial en terminal, esto porque si Arduino IDE está haciendo uso del puerto, en terminal no se podrán observar los valores.
\end{itemize}

%% realizarse en función de lo descrito en la sección de análisis de datos