% introduccion/resumen

%% consiste en el resumen del desarrollo y las conclusiones más importantes

%resumen
Con esta práctica, se busca desarrollar un sismógrafo digital utilizando la placa STM32F429 Discovery y la biblioteca libopencm3; con el cual se pueda registrar y analizar las oscilaciones de un determinado lugar, como por ejemplo la escuela de ingeniería eléctrica. Este dispositivo va a capturar datos del movimiento a través de un giroscopio, se puede establecer una comunicación serial, tener monitorización del nivel de batería, y visualizar los datos en un LCD. Adicionalmente, se trabaja con un script en Python para la transmisión de datos a un dashboard de ThingsBoard. \\

%conclusion
Con ayuda de los ejemplos proporcionados en el repositorio de la librería libopencm3-examples, se logra habilitar los LEDs, la pantalla LCD, configurar los pines GPIO como entradas o salidas segpun corresponda, y la habilitación del giroscopio incorporado en la placa. A través de los códigos de ejemplo proporcionados por la librería, se logra escribir texto para visualizar los resultados en la pantalla, así como activar la comunicación serial al presionar el botón que viene en la placa del microcontrolador. A partir de la comunicación serial, cuando se encuentra habilitada, se puede recibir los datos en el dashboard de Thingsboard con los widgets que facilitan la interpretación de los datos obtenidos del giroscopio.  \\

%link github 
En el siguiente link se encuentra el repositorio de trabajo: \url{https://github.com/NisseUR/IE-0624_Lab4}

